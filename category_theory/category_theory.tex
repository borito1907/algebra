\documentclass{article}
\usepackage{geometry}
\usepackage{graphicx} % Required for inserting images
\usepackage{amsmath, amsthm, amssymb}
\usepackage{mathtools}
\usepackage{parskip}
\newgeometry{vmargin={15mm}, hmargin={24mm,34mm}}
\theoremstyle{definition} 
\newtheorem{definition}{Definition}

\newtheorem{theorem}{Theorem}[section]
\newtheorem{lemma}[theorem]{Lemma}
\newtheorem{corollary}{Corollary}[theorem]
\newtheorem{proposition}[theorem]{Proposition}
\newtheorem{example}[theorem]{Example}

\newcommand{\N}{\mathbb{N}}
\newcommand{\Z}{\mathbb{Z}}
\newcommand{\R}{\mathbb{R}}
\newcommand{\Q}{\mathbb{Q}}
\newcommand{\catname}[1]{\mathbf{#1}}
\newcommand{\mor}{\text{Mor}}
\newcommand{\arr}{\text{Arr}}
\newcommand{\obj}{\text{Obj}}


\title{Category Theory}
\author{Boran Erol}

\begin{document}

\maketitle

\section{Introduction and Motivation}

Here's the theme of category theory according to Richard E. Borcherds: ignore
the internal structure of objects and analyze the morphisms between them.

Categories can be seen as:

\begin{enumerate}
    \item universes categorizing mathematical objects
    \item mathematical objects in their own right
\end{enumerate}

Category theory frames a possible template for any mathematical theory.

The purview of category theory is mathematical analogy. Many properties
of mathematical systems can be unified and simplified by a presentation
with diagrams of arrows.

Category theoretical proofs are often very elegant.

\newpage

\section{Definitions}

\begin{definition}[Serge Lang]
    A \textbf{category $C$} consists of a collection of objects $Obj(C)$ and for any objects
    $A,B \in Obj(C)$ we associate a set $Hom(A,B)$ called the \textbf{set of morphisms of $A$
    into $B$}. 
    
    Lastly, with any three objects $A,B,C$, we have the \textbf{law of composition},
    which is a map $Hom(A,B) \times Hom(B,C) \xrightarrow{} Hom(A,C)$ satisfying

    \begin{enumerate}
        \item $A \neq A^{\prime} \lor B \neq B^{\prime} \implies Hom(A,B) \cap Hom(A^{\prime},B^{\prime}) = \varnothing$
        Moreover, $A = A^{\prime} \land B = B^{\prime} \implies Hom(A,B) = Hom(A^{\prime}, B^{\prime})$.
        \item 
    \end{enumerate}
\end{definition}

$Hom(A,B)$ can be the empty set. A category can be totally disconnected in the sense
that $\forall A,B \in Obj(C): A \neq B \implies Hom(A,B) = \varnothing$.

It follows from our axioms that for any object $A$, $End(A)$ is a monoid under composition
and $Aut(A)$ is a group under composition.

\newpage

\subsection{Examples of Categories}

\begin{example}
    The \textbf{trivial category} has no objects and therefore no arrows.
\end{example}

\begin{example}
    We can have another category with a single object and a single identity mapping.
    In fact, there's a dictionary between categories with a single object and monoids
    from set theory.
\end{example}

\begin{example}
    Drawing random directed graphs will often define categories where the
    nodes are objects and the arrows are homomorphisms. However, you have to
    ensure that the properties of this directed graph obeys the axioms of 
    category theory.
\end{example}

\begin{example}
    Let $G$ be a group. We're going to construct a category $C$ out of this single
    group $G$. Let $Obj(C)$ be a singleton set. We don't actually care about the
    contents of this set. The data of the category is going to be fully captured
    using the homomorphisms.

    Let the homomorphisms of this single object $A$ be $G$, and define the
    composition law using the group product. 
\end{example}

\begin{example}
    Let $X$ be a set and $\leq$ be a preorder on $X$.
\end{example}

This might give a different lens for looking at categories. You can think of the most basic
category as a preorder. Then, regular categories are "thick" categories where there are many
possible relations (or "proofs") among different objects.

We can use the morphisms of an existing category and create new
categories. 

\newpage

\begin{definition}
    A category is \textbf{small} if it has only a set's worth of arrows.
\end{definition}

\begin{definition}
    A category is \textbf{locally small} if it has only a set's worth of 
    arrows between every pair of objects. In this case, the set is denoted
    with $C(X,Y)$ or $\hom(X,Y)$.
\end{definition}

\begin{definition}
    A category is a \textbf{thin category} if 

    \[ \forall f \in Arr(C): \lvert Arr(C) \rvert \leq 1\]
\end{definition}

\begin{definition}
    Let $A,B \in Obj(C)$ and $f \in Mor(A,B)$. $f$ is an \textbf{epimorphism}
    if $\forall C \in \obj(C): $
\end{definition}

\begin{definition}
    An object $I$ in a category $C$ is called \textbf{initial} if it admits
    a unique morphism to every $x \in C$. In other words, $Mor_{C}(I,X)$
    is a singleton for every $X \in C$.
\end{definition}

\begin{definition}
    An object $T$ in a category $C$ is called \textbf{terminal} if it admits
    a unique morphism from every $x \in C$. In other words, $Mor_{C}(X,T)$
    is a singleton for every $X \in C$.
\end{definition}

\begin{example}
    The trivial group is initial and terminal in the category of groups.
\end{example}

\begin{example}
    $\Z$ is the initial object in the category of rings.
\end{example}

\begin{definition}
    $f \in Mor(A,B)$ is an isomorphism if $\exists g \in Mor(B,A)$ such that

    \[ g \circ f = id_{a} \]
    \[ f \circ g = id_{b} \]
\end{definition}

Given a category $C$, you can reverse all the arrows and produce a new category.

\newpage

Notice that the definitions of category theory doesn't allow us
to peek inside the objects and use their structure. This makes
certain definitions difficult. For example, how does one define
injectivity in category theory? We can't consider "elements of
objects".

Instead, we define it as follows:

\newpage

\section{Kleisli Category}

You can understand Kleisli categories by thinking about adding a global log
to your functions and functional programming.

Milewski thinks that the difficulty people face when trying to understand monads 
is because they adopt an imperative programming approach, where a function takes
an object as an argument and returns another object. Instead, if you take the functional
programming approach where you are composing certain functions, monads just correspond
to composing with some flair.



\newpage

\section{Category Theory and Programming}

Operational semantics vs. denotational semantics

In math, all functions are pure. In programming, you have states (functions are not pure).

Functions have directionality, and this is implicit in category theory.



\newpage

\section{The Set Theoretical Elephant in the Room}

Consider the category of sets. Notice that the objects of this
category is the set of all sets! This flies in the face of Russels'
Paradox. In fact, finding a set-theoretical foundation for 
category theory is a challenge.

Here are some solutions:

\begin{enumerate}
    \item Bound the size of objects by some cardinal $\kappa$.
    \item Use "classes".
    \item Grothendiech universes
\end{enumerate}

We're going to ignore this problem for now, since this doesn't
really represent an issue until we go deep into category theory.



\newpage

\section{Functors}

Somewhat ironically, functors seem to be defined before categories.

The first such notion comes from considering the homology
groups of topological spaces.

Functors convert one type of object to another type of object.

A functor may describe the equivalence of categories: finite vector spaces and linear
maps are equivalent to the natural numbers and matrices.


\end{document}
