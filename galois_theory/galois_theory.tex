\documentclass{article}
\usepackage{geometry}
\usepackage{graphicx} % Required for inserting images
\usepackage{amsmath, amsthm, amssymb}
\usepackage{parskip}
\usepackage{xr}
\usepackage{amsmath}
\usepackage{amssymb}
\newgeometry{vmargin={15mm}, hmargin={24mm,34mm}}
\theoremstyle{definition} 
\newtheorem{definition}{Definition}

\newtheorem{theorem}{Theorem}[section]
\newtheorem{lemma}[theorem]{Lemma}
\newtheorem{corollary}{Corollary}[theorem]
\newtheorem{proposition}[theorem]{Proposition}
\newtheorem{example}[theorem]{Example}

\newcommand{\N}{\mathbb{N}}
\newcommand{\Z}{\mathbb{Z}}
\newcommand{\R}{\mathbb{R}}
\newcommand{\Q}{\mathbb{Q}}
\newcommand{\C}{\mathbb{C}}
\newcommand{\Nil}{\text{Nil}}
\newcommand{\gauss}{\Z[i]}

\title{Galois Theory}
\date{April 2024}
\author{Boran Erol}

\begin{document}

\maketitle

\section{Resources}

\begin{itemize}
    \item The Basic Graduate Year In Algebra
    \item Richard Elman, Lectures on Abstract Algebra, Chapters 11 and 12
\end{itemize}

\section{Definitions and Basic Properties}

\begin{definition}
    A \textbf{Fermat prime} is $f_{n} = 2^{2^{n}} + 1$ such that $f_{n}$ is prime.
\end{definition}

$f_{5}$ is not prime. $f_{1},f_{2},f_{3},f_{4}$ are prime. It's unknown whether we have
infinitely many non Fermat primes or Fermat primes. It's an extremely difficult number
theoretical problem.

If $n = 2^{n}p_{1}...p_{k}$ where $p_{i}$ is a Fermat prime, we can construct an $n$-gon.

\subsection{Applications of Galois Theory}

Langlands Program: Galois groups of extensions of $\Q$ have "something to do" with modular forms.
Proving bits of "something to do" leads to striking results. This is also used in the proof
of Fermat's Last Theorem.

The Galois group turns out to be related to the fundamental group in algebraic topology.
Field extensions end up corresponding to covering spaces. Algebraic closures end
up corresponding to universal covering spaces.

Inverse Problem: Given a finite group $G$, is there a Galois extension $K$ of $\Q$ such that Galois Group
of $K$ over $\Q$ is $G$? This is an open problem though it's been proven for specific types of groups such
as solvable and Abelian groups.

\newpage

\section{Classical Straightedge and Compass Constructions}


Construction by ruler and compass. You can't measure.

The ruler just allows you to draw straight lines between two points on the plane.

The compass allows you to draw a circle with some radius $r$.

Euclid

Here are some problems they considered:
\begin{enumerate}
    \item Constructing a sqaure with the area of a circle
    \item Trisecting an angle
    \item Constructing a regular polygon
\end{enumerate}

\newpage

\section{Finite Fields}



\end{document}
