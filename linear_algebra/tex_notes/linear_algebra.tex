\documentclass{article}
\usepackage{geometry}
\usepackage{graphicx} % Required for inserting images
\usepackage{amsmath, amsthm, amssymb}
\usepackage{parskip}
\usepackage{xr}
\usepackage{amsmath}
\usepackage{amssymb}
\newgeometry{vmargin={15mm}, hmargin={24mm,34mm}}
\theoremstyle{definition} 
\newtheorem{definition}{Definition}

\newtheorem{theorem}{Theorem}[section]
\newtheorem{lemma}[theorem]{Lemma}
\newtheorem{corollary}{Corollary}[theorem]
\newtheorem{proposition}[theorem]{Proposition}
\newtheorem{example}[theorem]{Example}

\newcommand{\N}{\mathbb{N}}
\newcommand{\Z}{\mathbb{Z}}
\newcommand{\R}{\mathbb{R}}
\newcommand{\Q}{\mathbb{Q}}
\newcommand{\C}{\mathbb{C}}
\newcommand{\Nil}{\text{Nil}}

\title{Linear Algebra}
\date{March 2024}
\author{Boran Erol}

\begin{document}

\maketitle

\section{Basic Definitions, Examples and Theorems}

Here are some examples of vector spaces:

\begin{example}
    $V = F[x]$ is a vector space over $F$ with basis $1,x,x^{2},...$.
\end{example}

\begin{example}
    Let $V$ be the space of real-valued functions on $[a,b]$ where $a < b$.
\end{example}

\begin{example}
    Let $V$ be the space of continuous real-valued functions on $[a,b]$ where $a < b$.
\end{example}

\begin{example}
    Let $F$ be a field and $f(x) = x^{n} + a_{n-1}x^{n-1} + ... + a_{1}x + a_{0} \in F[x]$.
    Consider $F[x]/(f(x))$. This is an $n$-dimensional vector spaces over $F$.

    \textbf{Expound on this using the Euclidean algorithm.}
\end{example}

\begin{theorem}
    Let $V$ be an $n$ dimensional vector space over $F$. Then, $V \cong F^{n}$.
\end{theorem}

\begin{theorem}
    Let $V$ be a vector space over $F$ and $W$ be a subspace of $V$. Then, $V/W$ is a vector space
    with $\dim(V) = \dim(W) + \dim(V/W)$.
\end{theorem}

\begin{corollary}
    Let $\phi: U \xrightarrow{} V$ be a linear transformation of vector spaces over $F$. Then, 
    \[\dim V = \dim(\ker \phi) + \dim(\phi(V))\].
\end{corollary}

\subsection{Exercises}

\begin{lemma}
    Let $V$ be a finite dimensional vector space and $\phi \in L(V)$.
    Then, \[\exists m \in \Z: Im(\phi^{m}) \cap \ker(\phi^{m}) = \{0\}\].
\end{lemma}
\begin{proof}
    
\end{proof}

\begin{lemma}
    Let $V$ be an $n$ dimensional vector space and $\phi \in L(V)$ such that $\phi^{2} = 0$.
    Then, $Im(\phi) \subseteq \ker(\phi)$. Therefore, rank$(\phi) \leq n/2$.
\end{lemma}
\begin{proof}
    $Im(\phi) \subseteq \ker(\phi)$ is easy: if $x$ in the image
    doesn't map to 0, $\phi^{2} \neq 0$. rank$(\phi) \leq n/2$
    is an immediate consequence of Rank-Nullity.
\end{proof}

\section{Diagonalization}

\begin{lemma}
    Every linear operator on a finite-dimensional complex vector space
    has an eigenvalue and a corresponding eigenvector.
\end{lemma}
\begin{proof}
    In Sheldon Axler's Linear Algebra Done Right. Basically uses the 
    fact that $\C$ is algebraically closed.
\end{proof}

\begin{definition}
    Let $A \in M_{n}(F)$. The \textbf{characteristic polynomial of A}
    is \[\det{(xI_{n} -A)} = x^{n} + ... + (-1)^{n} \cdot \det{A}\]

\end{definition}

Notice that this is a monic polynomial and the degree of the polynomial
is the dimension of the vector space $V$.

\newpage

\section{Dual Spaces}

\begin{lemma}
    Let $V$ be a finite dimensional vector spaces and $f_{1},...,f_{k}$ be
    elements of $V^{*}$ with kernels $N_{1},...,N_{k}$. Prove that $f \in V^{*}$ is a linear combination
    of $f_{1},...,f_{k}$ if and only if $N := ker(f)$ contains $\bigcap_{i = 1}^{k} N_{i}$.
\end{lemma}
\begin{proof}
    The forward implication is immediate. For the reverse implication, let $f \in V^{*}$
    such that 
\end{proof}

\newpage

\section{Determinants}

\begin{lemma}
    Let $A$ be an $n \times n$ matrix with column vectors $a_{1},...,a_{n}$.
    Find the determinant of the matrix with column vectors $a_{1} + a_{2},a_{2} + a_{3},...,a_{n} + a_{1}$.
\end{lemma}


\newpage

\section{Positive Definite Matrices}

\begin{lemma}
    Let $A,B$ be positive-definite matrices such that $A \leq B$, i.e. $B-A$ is positive-definite.
    Prove then that $B^{-1} \leq A_{-1}$.
\end{lemma}
\begin{proof}
    Recall that inverses of positive definite matrices are positive definite.


\end{proof}

\newpage

\section{Similar Matrices and Canonical Forms}

\begin{definition}
    Let $A,B \in M_{n}(F)$. $A$ and $B$ are \textbf{similar} if 
    there's an invertible $M \in M_{n}(F)$ such that $A = MBM^{-1}$.
\end{definition}

Similarity is an equivalence relation.

Similar matrices have the same eigenvalues. The converse is not true. 

\newpage

\section{Inner Products}

Orthonormal bases are useful in the sense that they allow us to compute the coordinates of 
vectors and inner products of vectors in an efficient fashion.

For example, let $b_{1},...,b_{n}$ be an orthonormal basis for $\R^{n}$.

Let $c = \sum_{i = 1}^{n} \alpha_{i}b_{i}$ and $c^{\prime} = \sum_{i = 1}^{n} \alpha^{\prime}_{i}b_{i}$.

Then, $\langle c, c^{\prime} \rangle = \sum_{i = 1}^{n} \alpha_{i}\alpha^{\prime}_{i}$.

\subsection{Exercises}

Let $V$ be the vector space of real-valued polynomials over $\R$ of degree at most 2.
Define the usual inner product:

\[ \langle p,q \rangle = \int_{0}^{1} p(t)q(t)dt\]

Find an orthogonal basis containing $p(x) = x$.

\textit{Use Gram-Schmidt}. One element can be $1 - \frac{3}{2}x$.

\newpage

\begin{lemma}
    Let $V$ be a finite-dimensional complex inner product space.
    Then, there's no invertible linear operator $T$ on $V$ such that $\forall v \in V: \langle Tv, v \rangle = 0$.
\end{lemma}
\begin{proof}
    Linear operators on $V$ have at least one eigenvalue: use this to reach a contradiction.
\end{proof}

However, this is not true for $\R$. Just consider the 90 degree rotation on $\R^{2}$
with the Euclidean norm.


\newpage

\section{Various Exercises}

\begin{lemma}
    Let $f$ be a linear operator in a vector space V over $\R$ such that $\forall v \in V: f(f(v)) = -v$. V has the structure of a vector space over $\C$ such that $\forall v \in V: f(v) = iv$.  
\end{lemma}
\begin{proof}
    
\end{proof}

\end{document}
