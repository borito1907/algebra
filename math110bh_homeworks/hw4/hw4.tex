\documentclass{article}
\usepackage{geometry}
\usepackage{graphicx} % Required for inserting images
\usepackage{amsmath, amsthm, amssymb}
\usepackage{parskip}
\newgeometry{vmargin={15mm}, hmargin={24mm,34mm}}
\theoremstyle{definition} 
\newtheorem{definition}{Definition}

\newtheorem{theorem}{Theorem}[section]
\newtheorem{lemma}[theorem]{Lemma}
\newtheorem{corollary}{Corollary}[theorem]


\title{MATH110BH Homework 4}
\date{February 2024}
\author{Boran Erol}

\begin{document}

\maketitle

\section{Problem 1}

\begin{lemma}
    Let $R = \mathbb{Z}[\sqrt{-5}]$. The ideal $I$ generated by $2$ and $1 + \sqrt(5)$ is not principal.
\end{lemma}
\begin{proof}
    Since every multiple $2$ and $1 + \sqrt{5}$ has even norm, $3 \notin I$. Therefore, $I \neq R$.

    By contradiction, assume $d \in R$ such that $(d) = (2, 1 + \sqrt{5})$. Then, $d \mid 2$ and $d \mid 1 + \sqrt{5}$. Then, $N(d) \mid 4$ and $N(d) \mid 6$. Then, $N(d) = 1$ or $N(d) = 2$. Since no element of $R$ satisfies $N(d) = 2$, $N(d) = 1$. Then, $d$ is a unit, which is a contradiction.
\end{proof}

\section{Problem 2}

\begin{lemma}
    $\mathbb{Z}[\sqrt{5}]$ is not a PID.
\end{lemma}
\begin{proof}
    Notice that $4 = 2 \times 2 = (1 + \sqrt{5})(1 - \sqrt{5})$. Notice that $\mathbb{Z}[\sqrt{5}]$ doesn't have elements with norm equal to $2$ or $3$. Therefore, $2$, $1 + \sqrt{5}$ and $1 - \sqrt{5}$ are all irreducible. Thus, $\mathbb{Z}[\sqrt{5}]$ is not a UFD, so it can't be a PID.
\end{proof}

\section{Problem 3}

\begin{lemma}
    Let $p$ be a prime number such that $p \equiv 3 \pmod{4}$. Then, $p$ is prime in $\mathbb{Z}[i]$.
\end{lemma}

\begin{proof}
    Notice $\mathbb{Z}[i]/(p) \simeq \mathbb{Z}[x]/(x^{2} + 1)/(p) \simeq \mathbb{Z}[x]/(p)/(x^{2} + 1) \simeq \mathbb{Z}/p\mathbb{Z}[x]/(x^{2}+1)$.If $\mathbb{Z}/p\mathbb{Z}[x]/(x^{2}+1)$ is a field, $(p)$ is maximal in $\mathbb{Z}[i]$ and therefore $(p)$ is prime in $\mathbb{Z}[i]$. Thus, it suffices to show that $\mathbb{Z}/p\mathbb{Z}[x]/(x^{2}+1)$ is a field. Since $\mathbb{Z}/p\mathbb{Z}$ is a field, $\mathbb{Z}/p\mathbb{Z}[x]$ is a PID. Therefore, it suffices to check that $x^{2} + 1$ is irreducible in $\mathbb{Z}/p\mathbb{Z}[x]$. This holds if and only if $x^{2}$ is a quadratic residue modulo $p$, which is true if and only if $p \equiv 1 \pmod 4$.
\end{proof}

\section{Problem 4}

\begin{lemma}
    Let $p$ be a prime number such that $p \equiv 1 \pmod{4}$. Then, $p$ is not prime in $\mathbb{Z}[i]$.
\end{lemma}
\begin{proof}
    Since $p \equiv 1 \pmod{4}$, $\exists x \in \mathbb{Z}: x^{2} \equiv -1 \pmod{p} \implies p \mid x^{2} + 1 = (x - i)(x + i)$. Therefore, to prove that $p$ is not prime in $\mathbb{Z}[i]$, it suffices to show that $p \nmid x - i$ and $p \nmid x + i$. Suppose that $p(a + bi) = x \pm i$. Then, $pa \mid x \implies x \equiv 0 \pmod p$, which implies $p = x$ or $x = 0$, which is a contradiction.
\end{proof}

\begin{lemma}
    Let $p$ be a prime number such that $p \equiv 1 \pmod{4}$. Then, $\exists a,b \in \mathbb{Z}: p = a^{2} + b^{2}$.
\end{lemma}
\begin{proof}
    Since $p$ is not prime in $\mathbb{Z}[i]$, there is some non-unit $x + yi$ with $x,y \in \mathbb{Z}$ that properly divides $p$. Then, the norm of $x + yi$ also properly divides $p$'s norm. In other words $x^{2} + y{2}$ properly divides $p^{2}$. Then, $x^{2} + y^{2} = 1$ or $x^{2} + y^2 = p$. $x^{2} + y^{2} = 1$ can't be true since $x + yi$ is not a unit. Therefore, $x^{2} + y^{2} = p$.
\end{proof}

\section{Problem 5}

\begin{lemma}
    Let $R$ be a PID and let $p$ be a prime element of $R$. Then, $pR$ is a maximal ideal.
\end{lemma}
\begin{proof}
    Suppose $p$ is a prime element of $R$. Then, the ideal $pR$ is maximal in the set of all principal ideals in $R$. Since $R$ is a PID, $pR$ is just a maximal ideal.
\end{proof}

\section{Problem 6}

\begin{lemma}
    Let $S,T$ be Noetherian rings and let $R = S \times T$ be the product ring where addition and multiplication are defined component-wise. Then, $R$ is also Noetherian.
\end{lemma}
\begin{proof}
    Recall from a previous homework exercise that every ideal of $R$ corresponds to $I \times J$ where $I$ is an ideal of $S$ and $J$ is an ideal of $T$. Since both $I$ and $J$ are finitely generated, $I \times J$ is also finitely generated. Thus, every ideal of $R$ is finitely generated.
\end{proof}

\section{Problem 7}

\begin{definition}
    An integral domain $R$ is called a \textbf{Bezout domain} if every ideal generated by two elements is a principal ideal.
\end{definition}

\begin{lemma}
    Let $R$ be ring. $R$ is a UFD if and only if $R$ is a Noetherian Bezout domain.
\end{lemma}
\begin{proof}
    The forward implication is trivial. Let $R$ be a PID. Then, every ideal in $R$ is finitely generated since it's principal. This holds for ideals generated by two elements as well. Therefore, $R$ is Noetherian and a Bezout domain.

    Let's now prove the converse of the statement. Let $R$ be a Noetherian Bezout domain and $I$ be an ideal of $R$. Since $R$ is Noetherian, $I$ is generated by finitely many elements $a_{1},...,a_{n}$. We'll now induct on $n$ to show that $I$ is a principal ideal. The case with $n = 1$ is trivial and the case with $n = 2$ follows from the fact that $R$ is a Bezout domain. Now, assume the statement holds for some $n \in \mathbb{N}$ and let $I$ be an ideal generated by $a_{1},a_{2},...,a_{n},a_{n+1}$. By the inductive hypothesis, the ideal generated by $a_{1},a_{2},...,a_{n}$ is principal and thus there exists some $\alpha \in R$ that generates it. Then, $I = (\alpha, a_{n+1})$. Since $R$ is a Bezout domain, it follows that $I$ is principal and we conclude our proof.
\end{proof}

\section{Problem 8}

\begin{lemma}
    Let $R_{1} \subset R_{2} \subset ...$ be a chain of countably many subrings of a ring $R$ such that $R = \bigcup R_{i}$. Suppose that all the $R_{i}$ are UFDs and any prime element in every $R_{i}$ is prime in $R_{i+1}$. Then, R is a UFD.
\end{lemma}
\begin{proof}
    Let $r \in R$. Then, $r \in R_{i}$ for some $i \in \mathbb{N}$. Then, $r = c_{1}\times ... \times c{n}$ where $c_{i}$ is an irreducible element of $R_{i}$. Therefore, it suffices to show that $c_{i}$ is irreducible in $R$. Let $c_{i} = uv$ for some $u,v \in R$. Then, $u,v \in R_{j}$ for some $j$ by taking the maximum. Then, $u$ or $v$ is a unit in $R_{j}$. Since a unit in a smaller ring is a unit in the larger ring, $u$ or $v$ is a unit in $R$. Also notice that $c_{i}$ doesn't become a unit in $R$ since that would imply that it's a unit in some $R_{j}$, which contradicts our assumption that irreducibles stay irreducible.


    Now, suppose by contradiction that factorization is not unique. Then, there is some $x \in R$ that has two different prime factorizations, i.e. $x = a_{1}a_{2}...a_{n} = b_{1}b_{2}...b_{m}$. Since $a_{i}$ and $b_{j}$ are all in $R$, they are all contained in some $R_{i}$ and $R_{j}$. By taking the maximum of all of these indices, we get some $R_{i}$ such that all $a_{i}$ and $b_{i}$ are in $R_{i}$ and are prime in $R_{i}$. However, this contradicts the fact that $R_{i}$ is a UFD.
\end{proof}

\section{Problem 9}

\begin{lemma}
    The polynomial ring $\mathbb{Z}[x_{1},x_{2},...]$ in countably many variables is a UFD but not a Noetherian ring.
\end{lemma}
\begin{proof}
    Notice that $\mathbb{Z}[x_{1},x_{2},...] = \bigcup R_{n}$ where $R_{n} = \mathbb{Z}[x_{1},x_{2},...,x_{n}]$. Then, $R_{i} \subset R_{i+1}$ and $f$ prime in $R_{i}$ $\implies$ $x$ is prime in $R_{i+1}$. Thus, by Problem 9, $\mathbb{Z}[x_{1},x_{2},...]$ is a UFD.

    Now, let $I_{n}$ be the ideal generated by $(x_{1},...,x_{n})$. Clearly, $I_{n} \subset I_{n+1}$, but this increasing sequence of ideals never ends. Therefore, $\mathbb{Z}[x_{1},x_{2},...]$ is not Noetherian.
\end{proof}

\section{Problem 10}

Notice that the product of generators is a generator for the product of ideals.

\[ (2 + (1 + \sqrt{-5}))(3 + ( 1 + \sqrt{-5}) = (6 + (2 + 2\sqrt{-5}) + (3 + 3\sqrt{-5}) + 6)\]

Then, these four elements generate the product of ideals. Notice that the difference of the second and the third generator is precisely $1 + \sqrt{-5}$, therefore the ideal $1 + \sqrt{-5}$ is contained in the product ideal.

To see the reverse containment, notice that $1 + \sqrt{-5}$ divides all the generators. It divides the second and third generators trivially, and the first and fourth using its conjugate.
\end{document}
