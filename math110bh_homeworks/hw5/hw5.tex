\documentclass{article}
\usepackage{geometry}
\usepackage{graphicx} % Required for inserting images
\usepackage{amsmath, amsthm, amssymb}
\usepackage{parskip}
\newgeometry{vmargin={15mm}, hmargin={24mm,34mm}}
\theoremstyle{definition} 
\newtheorem{definition}{Definition}

\newtheorem{theorem}{Theorem}[section]
\newtheorem{lemma}[theorem]{Lemma}
\newtheorem{corollary}{Corollary}[theorem]
\def\gauss{\ensuremath\mathbb{Z}[i]\ }

\title{MATH110BH Homework 5}
\date{January 2024}
\author{Boran Erol}

\begin{document}

\maketitle

\section{Problem 1}

\begin{lemma}
    Let $F$ be a field. Then, $F[X]$ has infinitely many irreducible polynomials.
\end{lemma}
\begin{proof}
    This is the exact same proof as Euclid's proof that there are infinitely many prime numbers.

    Suppose not. Let $f_{1}, ..., f_{n}$ be the irreducible polynomials. Consider $g = (f_{1} \times f_{2} \times ... \times f_{n}) + 1$. Since $g$ is not irreducible, there's some $f_{i}$ that divides $g$. Then, $f_{i} \mid 1$, which is a contradiction.
\end{proof}

\section{Problem 2}

\begin{lemma}
    $\mathbb{Z}[i]/(1+i)\mathbb{Z}[i] \cong F_{2}$
\end{lemma}
\begin{proof}
    First of all, notice that $(1+i)(1+i) = 2i$ and $(1+i)(1-i) = 2$, so $(1+i)R$ includes all Gaussian integers with even coefficients.

    Consider the map $f: \gauss \to F_{2}$ defined by $a + bi \mapsto a + b \pmod 2$. This is clearly a group homomorphism. 

    Let's now prove that it is a ring homomorphism. Notice that $(a + bi)(c + di) = (ac - bd) + (ad + bc) = a(c+d) + b(c-d) = a(c+d) + b(c+d) = (a+b)(c+d)$, since $c+d = c-d$ in $F_{2}$. Thus, it's a ring homomorphism.

    By the first isomorphism theorem, it suffices to prove that $ker(f) = (1 + i)R$.

    Let $a + bi \in ker(f)$. There are two cases we need to handle:

    \begin{enumerate}
        \item Both $a,b$ are even.

            Then, $a + bi \in (1 + i)R$ by the initial discussion.
        \item Both $a,b$ are odd.
            Then, $a + bi = 2k + 1 + 2mi + i$. Since $2k + 2mi \in (1+i)R$, $a + bi \in (1 + i)R$.

    Now, let $a + bi \in (1 + i)R$. Then, $a + bi =(c + di)(1 + i) = c + ci + di -d $ for some $c,d \in \mathbb{Z}$. Notice that if $c,d$ are both even or odd, both coefficients are even so $a + bi \in ker(f)$. If one of them is odd and the other is even, both coefficients are even so $a + bi \in ker(f)$. 
    \end{enumerate}
\end{proof}

\section{Problem 3}

\begin{lemma}
    Let $f \in \mathbb{Q}[x]$. $f \in \mathbb{Z}[x]$ if and only if $Cont(f) \in \mathbb{Z}$.
\end{lemma}
\begin{proof}
    The forward implication is trivial. Let's prove the converse. Let $f = a_{n}x^{n} + a_{n-1}x^{n-1} + ... + a_{1}x + a_{0}\in \mathbb{Q}[x]$ and $m = min\{n : nf \in \mathbb{Z}[x]\}$. Then, $Cont(f) = \frac{1}{m} \gcd{(ma_{1},...,ma_{n})}$. If $Cont(f) \in \mathbb{Z}$, the greatest common divisor is a multiple of $m$. Then, $\frac{m a_{i}}{m}$ is an integer for every $i$, so $f \in \mathbb{Z}[x]$.
\end{proof}

\begin{lemma}
    Let $f,g \in \mathbb{Q}[x]$ with $fg \in \mathbb{Z}[x]$. Then, $\exists a \in \mathbb{Q}^{\times}: af \in \mathbb{Z}[x] \land a^{-1}g \in \mathbb{Z}[x]$.
\end{lemma}
\begin{proof}
    Let $Cont(f) = \frac{p_{1}}{q_{1}}$ and $Cont(g) = \frac{p_{2}}{q_{2}}$ be such that $p_{i}$ and $q_{i}$ are coprime. Since $fg \in \mathbb{Z}[x]$, $Cont(f)Cont(g) = Cont(fg) \in \mathbb{Z}$. Let $a = \frac{p_{2}}{q_{2}}$. Then, $Cont(af) \in \mathbb{Z}$ and $Cont(a^{-1}g) = 1$, so $af \in \mathbb{Z}[x]$ and $a^{-1}g \in \mathbb{Z}[x]$. We conclude the proof using the lemma above.
\end{proof}

\section{Problem 4}

Let $F$ be a field. 
Let $R$ be the set of polynomials in $F[X]$ whose $X$-coefficient is 0. This set is clearly closed under addition and multiplication. $f = 1$ is also in $R$, so $R$ is a subring of $F[X]$. Moreover, notice that $X^{2}$ and $X^{3}$ are irreducibles in $R$ since $X \notin R$. Moreover, $X^{6} = (X^{2})^{3} = (X^{3})^{2}$ so $X^{6}$ has two different factorizations.

\section{Problem 5}

Constant polynomials aren't irreducible by definition. Both $x$ and $x + 1$ are irreducible since every polynomial of degree $1$ is irreducible. $x^{2} + x + 1$ is the only polynomial of degree $2$ without a root so it is irreducible. Similarly, $x^{3} + x + 1$ and $x^{3} + x^{2} + 1$ are the only cubic polynomials without roots, so they're irreducible. As for fourth degree polynomials, notice that every polynomial should have the following form: 

\[ x^{4} + ax^{3} + bx^{2} +cx + 1\]

since otherwise $0$ is a root. Moreover, $a + b + c$ needs to be odd since otherwise $1$ is a root. Since $f$ shouldn't have roots, it also can't have a linear factor. Therefore, we only need to consider the square of irreducible polynomials of degree $2$, of which there's one. Since $(x^{2} + x + 1)^2 = x^{4} + x^{2} + 1$, the irreducible polynomials are $x^{4} + x^{3} + 1$ and $x^{4} + x + 1$.

\section{Problem 6}

\begin{lemma}
    Let $f \in \mathbb{Z}[x]$ and $a,b \in \mathbb{Z}$. Then, $a - b \mid f(a) - f(b)$.
\end{lemma}
\begin{proof}
    We'll induct on the degree of $f$.
    The statement is trivially true when $deg(f) = 0$ since every integer divides $0$. Similarly, the statement is clearly true when $deg(f) = 1$ since $a - b \mid k(a - b)$. Now, assume the statement is true for some $n \in \mathbb{N}$. Let $f = a_{n+1}x^{n+1} + a_{n}x^{n} + ... + a_{1}x + a_{0}$. Notice that $g = a_{n}x^{n} + ... + a_{1}x + a_{0}$ is a polynomial of degree $n$. Also notice that 

    \[ f(a) - f(b) = a_{n}(a^{n} - b^{n}) + (g(a) - g(b))\]
    
    
    By the inductive hypothesis, $a - b \mid g(a) - g(b)$. Since $a - b \mid a^{n} - b^{n}$, $a - b \mid f(a) - f(b)$.
\end{proof}


\section{Problem 7}

Since $\mathbb{Z}[X,Y] = \mathbb{Z}[X][Y]$, we can consider $y^{n} + (x^{n} - 1)$ as a polynomial with coefficients $1$ and $(x^{n} - 1)$. Notice that $x - 1$ is an irreducible in $Z[X,Y]$. Since $\mathbb{Z}[X,Y]$ is a UFD, $x - 1$ is also a prime. Moreover, $x - 1 \mid x^{n} - 1$ and $x - 1 \nmid 1$. However, $(x-1)^2 \nmid x^{n} - 1$. Then, by Eisenstein's Criterion, $y^{n} + (x^{n} - 1)$ is irreducible.


\section{Problem 8}

This is a special case of the rational root theorem. Let $f = x^{n} + a_{n-1}x^{n-1} + ... + a_{1}x + a_{0}$ and assume $a \in Q$ is a root of $f$. Let $a = \frac{p}{q}$ be the most simplified version of $a$. Then, \[(\frac{p}{q})^{n} + a_{n-1}\frac{p}{q}^{n-1} + ... + a_{1}\frac{p}{q} + a_{0} = 0\]

Multiplying by $q^{n}$ and rearranging gives

\[ -p^{n} = q(a_{0}q^{n-1} + a_{2}pq^{n-2} + ... + a_{n-1}p^{n-1})\]

Then, $q \mid p$. Since they're relatively prime, this produces $q = 1$.


\section{Problem 9}

Let $f = x^p -x$ be a polynomial in $(\mathbb{Z}/p\mathbb{Z})[x]$. By Fermat's Little Theorem, every non-zero value in $(\mathbb{Z}/p\mathbb{Z})$ is a root of $f$. Recall that every root produces a linear factor and that a polynomial has at most $deg(f)$ linear factors. Therefore, $f = x(x-1)(x-2)...(x-p+1)$. 


\section{Problem 10}

Notice that $x^{4} + 4 = (x^{2} + 2x + 2)(x^{2} - 2x + 2)$, so $x^{4} + 4$ is not irreducible. There are two ways to see this:

First, notice that $x^4 + 4 = x^4 + 4x^2 + 4 - 4x^2 = (x^2 + 2)^2 - (2x)^2 = (x^{2} + 2x + 2)(x^{2} - 2x + 2)$.

Another, more straightforward way to see this is to consider the complex roots of $x^{4}$. Since all complex roots have integer coefficients, the product of conjugate pairs is going to be in $Z[x]$, so $x^{4} + 4$ is reducible.

\end{document}
