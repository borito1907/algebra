\documentclass{article}
\usepackage{geometry}
\usepackage{graphicx} % Required for inserting images
\usepackage{amsmath, amsthm, amssymb}
\usepackage{parskip}
\newgeometry{vmargin={15mm}, hmargin={24mm,34mm}}
\theoremstyle{definition} 
\newtheorem{definition}{Definition}

\newtheorem{theorem}{Theorem}[section]
\newtheorem{lemma}[theorem]{Lemma}
\newtheorem{corollary}{Corollary}[theorem]
\def\gauss{\ensuremath\mathbb{Z}[i]}
\newcommand{\Z}{\mathbb{Z}}
\newcommand{\R}{\mathbb{R}}


\title{MATH110BH Homework 7}
\date{February 2024}
\author{Boran Erol}

\begin{document}

\maketitle

\section{Problem 1}



\begin{lemma}
    Let $R$ be a UFD, The intersection of two principal ideals $aR$ and $bR$ is a principal ideal generated by $lcm(a,b)$.
\end{lemma}
\begin{proof}
    Let $a = p_{1}^{\alpha_{1}}p_{2}^{\alpha_{2}}...p_{n}^{\alpha_{n}}$ and $b = p_{1}^{\beta_{1}}p_{2}^{\beta_{2}}...p_{n}^{\beta_{n}}$ where $\alpha_{i}$ and $\beta_{i}$ are non-negative integers. Let $m_{i} = \max \{ \alpha_{i}, \beta_{i} \}$.
    Let $c := lcm(a,b) = p_{1}^{m_{1}}...p_{n}^{m_{n}}$. We'll prove that $aR \cap bR = cR$.

    Notice that $aR \mid cR$ and $bR \mid cR$, so $cR \subseteq aR \cap bR$. Now, let $x \in aR \cap bR$. Notice that $p_{i}^{m_{i}} \mid x$, so $c \mid x \implies x \in cR$. Thus, $aR \cap bR \subseteq cR$ and we conclude the proof.
\end{proof}

\section{Problem 2}

Let $F$ be a field. Then, $R = F[x_{1},x_{2},...]$ is a Euclidean domain. Notice that $R$ is a free $R$-module since it's generated by $1$. Now, let $N$ be the submodule generated by $\{x_{1},x_{2},...\}$. Clearly, finitely many elements can't generate $N$ since every $x_{i}$ has to be in any generating set. Now, notice that if $x_{i}$ and $x_{j}$ are in some set $S$, $x_{i} \cdot x_{j} = x_{j} \cdot x_{i}$, so $S$ is not independent. Therefore, $N$ is not a free module.

Recall (2,x) not principal in Z[x] and use prev hw problem

\section{Problem 3}

\begin{lemma}
    Let $R$ be a PID and $M$ be a $R$-module generated by $n$ elements. Let $N$ be a submodule of $M$. Then, $N$ can also be generated by $n$ elements.
\end{lemma}
\begin{proof}
    Let $M$ be an $R$-module generated by $\{ x_{1}, ..., x_{n} \}$. Let $f: R^{n} \xrightarrow{} M$ defined by mapping $e_{i} \mapsto x_{i}$. This is a surjective module homomorphism. Let $P = f^{-1}(N)$. Since $N$ is a submodule of $M$, $P$ is a submodule of $R^{n}$. Then, by the theorem proven in class, $P$ is generated by at most $n$ elements. Then, let $\{y_{1},...,y_{m}\}$ be the generating set for $P$ where $m \leq n$. Then, $\{f(y_{1}),...,f(y_{m})\}$ is a generating set for $N$.
\end{proof}

\section{Problem 4}

\begin{lemma}
    The group $\mathbb{Z}^{n}$ can't be generated by $n - 1$ elements.
\end{lemma}
\begin{proof}
    $\mathbb{Z}^{n}$ has a standard basis consisting of $n$ elements. We showed in Problem 8 of Homework 6 that every free module generated by $m$ elements has a basis consisting of at most $m$ elements. In Problem 9, we showed that every two bases for a free finitely generated R-module have the same number of elements. Putting these two facts together, any generating set of $\mathbb{Z}^{n}$ should have at least $n$ elements.

    BRING IT DOWN TO THE EMPTY SET.

    USE Q INSTEAD.
\end{proof}

\section{Problem 5}

Let $M = \Z / 2 \Z$ and $N = \Z / 3 \Z$. Clearly, these aren't free $\Z / 6 \Z$ modules since $2 \cdot x = 0$ for any $x \in M$ and $3 \cdot x = 0$ for any $x \in N$. However, $(1,1)$ generates $M \bigoplus N$, so $M \bigoplus N$ is free (in fact, cyclic).

\section{Problem 6}

\begin{lemma}
    Let $R$ be a PID and $M$ be a torsion finitely generated R-module with the invariant factors $d_{1} \mid d_{2} \mid ... \mid d_{k}$. Let $I = \{a \in R: aM = 0\}$. Then, $I = d_{k}R$.
\end{lemma}
\begin{proof}
    We prove both inclusions. By the representation for finitely generated modules, we have that $M \cong R/d_{1}R \bigoplus ... \bigoplus R/d_{k}R$. Let $x \in I$ and $s \in R/d_{k}R$. Then, $xs \in d_{k}R$ for any $s \in R$ since $x \cdot (s + d_{k}R) = d_{k}R$ for any $s \in R$. Letting $s = 1$, $x \in d_{k}R$. Now, let $s \in d_{k}R$. Then, $d_{k} \mid s \implies d_{i} \mid s$ for any $i$. Then, $s \in d_{i}R$ for any $i$. Then, $s \cdot (x + d_{i}R) = d_{i}R$ for any $i$, so $s \in I$. We thus conclude the proof.
\end{proof}

\section{Problem 7}

Let $A$ be an Abelian group of order 300. By the primary decomposition theorem, $A \cong A_{1} \times A_{2} \times A_{3}$ where $\lvert A_{1} \rvert = 2^{2}$, $\lvert A_{2} \rvert = 3$ and $\lvert A_{3} \rvert = 5^{2}$. Notice that $A_{1} \cong C_{4}$ or $A_{1} \cong C_{2} \times C_{2}$, $A_{2} \cong C_{3}$ and $A_{3} \cong C_{25}$ or $A_{3} \cong C_{5} \times C_{5}$.

Therefore, we have the following 4 possible groups:

\[ C_{4} \times C_{3} \times C_{25}\]
\[ C_{2} \times C_{2} \times C_{3} \times C_{25}\]
\[ C_{4} \times C_{3} \times C_{5} \times C_{5}\]
\[ C_{2} \times C_{2} \times C_{3} \times C_{5} \times C_{5}\]


\section{Problem 10}

\begin{lemma}
    Let $R$ be a PID and $M$ be an R-module. $M$ is cyclic if and only if $M \cong R/(a)$ for some $a \in R$.
\end{lemma}
\begin{proof}
    Assume $M$ is cyclic. Then, there's some $x \in M$ such that $x$ generates $M$. Consider the module homomorphism $\phi_{x}: R \xrightarrow{} M$ given by $\phi(r) = rx$. Since $x$ generates $M$, $\phi$ is surjective. Since $R$ is a PID, $ker(\phi_{x}) = (a)$ for some $a \in R$.
    
    Then, by the first isomorphism theorem for modules, $M \cong R/(a)$. 

    For the converse, notice that $a$ is a generator for the module $R/(a)$.
\end{proof}

\begin{lemma}
    Let $M$ be a finitely generated torsion module over a PID $R$. $M$ is cyclic if and only if every two elementary divisors of $M$ are relatively prime.
\end{lemma}
\begin{proof}
    Assume $M$ is cyclic. Then, $M \cong R/(a)$. Since $R$ is a UFD, we have that a has a unique prime factorization. 

    \[ a = p_{1}^{\alpha_{1}}...p_{n}^{\alpha_{n}}\]

    Then, by the Chinese Remainder Theorem,

    \[ R/aR \cong R/p_{1}^{\alpha_{1}}R \oplus ... \oplus R/p_{n}^{\alpha_{n}}R \]

    Clearly, $p_{i}$ and $p_{j}$ are relatively prime. To see the converse, just apply the Chinese Remainder theorem again and see that $M \cong R/aR$ for some $a \in R$. 
    
\end{proof}
    
\end{document}
