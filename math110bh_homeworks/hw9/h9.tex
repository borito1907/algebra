\documentclass{article}
\usepackage{geometry}
\usepackage{graphicx} % Required for inserting images
\usepackage{amsmath, amsthm, amssymb}
\usepackage{parskip}
\newgeometry{vmargin={15mm}, hmargin={24mm,34mm}}
\theoremstyle{definition} 
\newtheorem{definition}{Definition}

\newtheorem{theorem}{Theorem}[section]
\newtheorem{lemma}[theorem]{Lemma}
\newtheorem{corollary}{Corollary}[theorem]
\def\gauss{\ensuremath\mathbb{Z}[i]}

\newcommand{\N}{\mathbb{N}}
\newcommand{\Z}{\mathbb{Z}}
\newcommand{\R}{\mathbb{R}}
\newcommand{\C}{\mathbb{C}}


\title{MATH110BH Homework 9}
\date{March 2024}
\author{Boran Erol}

\begin{document}

\maketitle

\section{Problem 6}

\begin{lemma}
    Two $2 \times 2$ non-scalar matrices are similar if and only if they have the same characteristic polynomials.
\end{lemma}
\begin{proof}
    If two matrices are similar, they have the same invariant factors, so they have the same characteristic
    polynomials. Let's now prove the converse.

    Let $A,B$ be two matrices with the same characteristic polynomial, $f$.
    
    Assume $f$ is quadratic. If $f$ has no roots, it's irreducible, there's only one IF decomposition,
    so the matrices are similar.

    If $f$ has the same repeated root, it's either diagonalizable (in which case it's scalar) or
    the invariant factor is $(x-a)^{2}$. If it's diagonalizable, it's a scalar matrix, so the only IF
    is $(x-a)^{2}$.

    If it has different roots, the only IF $(x-a)(x-b)$, so they're similar.

    If the characteristic polynomial is irreducible, there's only one IF decomposition and $A,B$ are similar. 

    We thus conclude the proof.
\end{proof}

\section{Problem 7}

\begin{lemma}
    Two $3 \times 3$ matrices are similar if and only if they have the same characteristic and the same
    minimal polynomials.
\end{lemma}
\begin{proof}
    The forward direction is trivial. Now, let $A,B$ matrices with the same characteristic and similar
    polynomials. To show that they're similar, it suffices to show that they have the same IF form.
    
    If the minimal polynomial has degree 3, $A,B$ have the same IF form.
    If the minimal polynomial has degree 2, the other invariant factor has degree 1. $A,B$ should have
    the same invariant factor since otherwise their characteristic polynomial would be different.
    If the minimal polynomial has degree 1, the IF form is the minimal polynomial repeated thrice,
    so $A,B$ have the same IF form.

    We thus conclude the proof.
\end{proof}

\section{Problem 8}

\begin{lemma}
    Let $A \in M_{n}(F)$. The minimal polynomial of $A$ has the same irreducible divisors
    as the characteristic polynomial of $A$.
\end{lemma}
\begin{proof}
    Since the minimal polynomial divides the characteristic polynomial, it suffices to show
    that every irreducible divisor of the characteristic polynomial divides the minimal polynomial.

    Let $f $ be an irreducible polynomial that divides the characteristic polynomial. Then, $f$ is
    also prime since $F[x]$ is a Euclidean domain. Then, $f$ divides one of the invariant factors by
    a simple inductive argument on the number of invariant factors. Since the minimal polynomial is
    the greatest invariant factor, $f$ also divides the minimal polynomial.
\end{proof}

\section{Problem 9}

\begin{lemma}
    Let $A \in M_{n}(F)$ be a nilpotent matrix. Then, the invariant factors of $A$
    are powers of $x$.
\end{lemma}
\begin{proof}
    If $A^{n} = 0$, the minimal polynomial of $A$ divides $x^{n}$. Then, the minimal
    polynomial is of the form $x^{m}$ for some $m < n$. Since the minimal polynomial
    is the largest IF, we're done. 
\end{proof}

Here's another rabbithole that would work but would take some time:

\begin{proof}
    If $A$ is nilpotent, then the companion matrices of $A$ also have to be nilpotent. We'll induct on $i$
    to show that $a_{i} = 0$. If $a_{0} \neq 0$, then the companion matrix has full rank and is an isomorphism,
    which contradicts its nilpotence.
\end{proof}

\begin{corollary}
    Let $A \in M_{n}(F)$ be a nilpotent matrix. Then, $A^{n} = 0$.
\end{corollary}
\begin{proof}
    Since the invariant factors of $A$ are powers of $x$ and the characteristic polynomial is the product
    of the invariant factors, the characteristic polynomial is $c_{A}(x) = x^{n}$. By Cayley-Hamilton, 
    $A^{n} = 0$.
\end{proof}

\section{Problem 10}

\begin{lemma}
    Every $n \times n$ matrix is similar to its transpose.
\end{lemma}
\begin{proof}
    We can use the fact that Jordan blocks are similar to their transposes to show this for algebraically closed
    fields. However, we'd like a general solution.

    Instead, we can use RCF. Notice that both $XI - A$ and $XI - A^{t}$ are similar to a diagonal matrix.
    In fact, by replacing every column operation in the computation of $XI - A$ with a row operation
    and vice versa, we can see that they're the same matrix. Therefore, they have the same RCF form,
    and are thus similar.
\end{proof}



\end{document}

