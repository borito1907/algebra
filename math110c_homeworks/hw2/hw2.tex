\documentclass{article}
\usepackage{geometry}
\usepackage{graphicx} % Required for inserting images
\usepackage{amsmath, amsthm, amssymb}
\usepackage{parskip}
\usepackage{xr}
\usepackage{amsmath}
\usepackage{amssymb}
\newgeometry{vmargin={15mm}, hmargin={24mm,34mm}}
\theoremstyle{definition} 
\newtheorem{definition}{Definition}

\newtheorem{theorem}{Theorem}[section]
\newtheorem{lemma}[theorem]{Lemma}
\newtheorem{corollary}{Corollary}[theorem]
\newtheorem{proposition}[theorem]{Proposition}
\newtheorem{example}[theorem]{Example}

\newcommand{\N}{\mathbb{N}}
\newcommand{\Z}{\mathbb{Z}}
\newcommand{\R}{\mathbb{R}}
\newcommand{\Q}{\mathbb{Q}}
\newcommand{\C}{\mathbb{C}}
\newcommand{\Nil}{\text{Nil}}
\newcommand{\gauss}{\Z[i]}

\title{MATH110C Homework 2}
\date{April 2024}
\author{Boran Erol}

\begin{document}

\maketitle

\section{Exercise 1}

\newpage

\section{Exercise 2}

\newpage

\section{Exercise 3}

\newpage

\section{Exercise 4}

\begin{lemma}
    The splitting field of $p(x) = x^{4} - 9$ is $\Q(\sqrt{3},i)$.
\end{lemma}
\begin{proof}
    Notice that the roots over $\C$ are $\pm \sqrt{3}, \pm i \sqrt{3}$.
    Therefore, $\Q(\sqrt{3},i)$ contains all roots of $p$. Conversely,
    any field $K$ that contains all four roots contains $\sqrt{3}$ and $i$
    since $i = \frac{\sqrt{3}i}{\sqrt{3}}$.
\end{proof}

\newpage

\section{Elman pg.307 Problem 1e}

$t^{6} + t^{3} + 1 = \Phi_{9}(t)$, so it is irreducible by Problem 7a.
Therefore, $[K: \Q] = 6$.

\newpage

\section{Elman pg.307 Problem 2b}

\begin{lemma}
    Let $K$ be the splitting field of $p(x) = x^{6} - 8$ over $\Q$.
    Then, $[K : \Q] = 4$.
\end{lemma}
\begin{proof}
    Note that $\sqrt{2}$ is a solution to $p$. Letting, $w = e^{\pi / 3 i}$,
    the roots of $p$ are $\sqrt{2}, \sqrt{2} w, sqrt{2} w^{2},...,\sqrt{2}w^{5}$.
    Notice that $w = \frac{1}{2} + i \frac{\sqrt{3}}{2}$ using Euler's formula.

    Thus, $K = \Q(\sqrt{2},\sqrt{-3})$. Notice that $\sqrt{-3}$ satisfies $x^{2} + 3$
    and $\sqrt{-3} \notin \Q(\sqrt{2})$. Thus, $[\Q(\sqrt{2},\sqrt{-3}): \Q] = [\Q(\sqrt{2},\sqrt{-3}): \Q(sqrt{2})][\Q(\sqrt{2}): \Q]
    = 2 \times 2 = 4$.
\end{proof}

\newpage

\section{Elman pg.308 Problem 5}


\newpage

\section{Elman pg.308 Problem 6}



\newpage

\section{Elman pg.308 Problem 7}


\end{document}
