\documentclass{article}
\usepackage{geometry}
\usepackage{graphicx} % Required for inserting images
\usepackage{amsmath, amsthm, amssymb}
\usepackage{mathtools}
\usepackage{parskip}
\newgeometry{vmargin={15mm}, hmargin={24mm,34mm}}
\theoremstyle{definition} 
\newtheorem{definition}{Definition}

\newtheorem{theorem}{Theorem}[section]
\newtheorem{lemma}[theorem]{Lemma}
\newtheorem{corollary}{Corollary}[theorem]
\newtheorem{proposition}[theorem]{Proposition}
\newtheorem{example}[theorem]{Example}

\newcommand{\N}{\mathbb{N}}
\newcommand{\Z}{\mathbb{Z}}
\newcommand{\R}{\mathbb{R}}
\newcommand{\Q}{\mathbb{Q}}
\newcommand{\Tor}{\text{Tor}}

\title{Module Theory}
\author{Boran Erol}

\begin{document}

\maketitle

Here's the basic problem in module theory:

Given a ring with certain nice properties, classify all modules over that ring.

This problem turns out to be extremely difficult. Therefore, we restrict to certain types of modules and try classifying those modules.

Modules are the representation objects for rings. They are, by definition, algebraic objects on which rings act.

\section{Definitions and Basic Properties}

\subsection{Definition and Characterization using the Endomorphism Ring}

\begin{definition}
    Let $R$ be a ring and $M$ be an Abelian group. $M$ is said to be a left R-module if
    \begin{enumerate}
        \item $(M,+)$ is an additive Abelian group.
        \item $\forall x \in M: 1 \cdot x = x$
        \item $\forall a,b \in R: \forall x \in M: a \cdot (b \cdot x) = (ab) \cdot x$
        \item $\forall a \in R: \forall x,y \in M: a(x + y) = ax + ay$
        \item $\forall a,b \in R: \forall x \in M: (a + b)x = ax + bx$
    \end{enumerate}
\end{definition}

Let $M$ be a left $R$-module. Fix $a \in R$. Notice that using Axiom 4, the map $f_{a}: M \xrightarrow{} M$ defined by $f_{a}(x) = ax$ is a group homomorphism. Now, consider $\phi: R \xrightarrow{} \text{End}(M)$ defined by $\phi(a) = f_{a}$. 

\begin{lemma}
    $\phi$ is a ring homomorphism.
\end{lemma}
\begin{proof}
    \[ \phi(a + b)(x) = f_{a+b}(x) = (a+b)x = ax + bx = f_{a}(x) + f_{b}(x) = \phi(a)(x) + \phi(b)(x)\]

    \[ \phi(ab)(x) = f_{ab}(x) = (ab)x = a(bx) = f_{a}(bx) = f_{a}(f_{b}(x)) = \phi(a)\phi(b)(x)\]    
\end{proof}

Similarly, given $\phi: R \xrightarrow{} End(M)$ a ring homomorphism, we can give $M$ a module structure.

\newpage

\subsection{Exercises}

\begin{lemma}
    Let $F$ be a field and $R = M_{n}(F)$. An element $A \in R$ is a zero divisor if and only if
    it's singular.
\end{lemma}

The units of $M_{n}(F)$ form the \textbf{general linear group $GL_{n}(F)$}.

\newpage

\section{Isomorphism Theorems for Modules}

\begin{theorem}
    Let $\phi: M \xrightarrow{} N$ be a module homomorphism. Then,

    \[ M/\ker(\phi) \cong N \]
\end{theorem}

\begin{theorem}
    Let $M$ be a module and $N,K$ be submodules. Then,

    \[ N/(N \cap K) \cong (N + K)/ K\]
\end{theorem}
\begin{proof}
    Consider $\phi: N \xrightarrow{} (N + K)/K$ defined by $\phi(n) = n + K$.
    $\ker(\phi) = N \cap K$, so we conclude the proof by the first isomorphism
    theorem.
\end{proof}

\begin{theorem}[Correspondence Theorem]
    Let $M,N$ be $R$-modules and $f: M \xrightarrow{} N$ be a surjective module homomorphism.
    Then, \[\Phi: \{ \text{submodules of M containing $\ker(f)$}\} \xrightarrow{} \{ \text{submodules of N}\} \]
    is a bijection given by $\Phi(K) = f(K)$.
\end{theorem}

In particular, considering $M = R$ as a module over itself and $N = I$ for some ideal $I$, 
this reproduces the correspondence theorem for ideals.

\newpage

\section{Endomorphism Rings of Modules}

Let $R$ be a ring and $M,N$ be $R$-modules. We'll examine $Hom_{R}(M,N)$.

This is an additive Abelian group using pointwise addition.

It's an $R$-module where the $R$-action is defined to be $r \cdot f(x) = rf(x)$.

Notice that you can't really endow it with a natural ring structure if $M \neq N$ since
function composition doesn't work. If $M = N$, though, this natural ring structure
works and is called the \textbf{endomorphism ring of the module $M$}.

\begin{lemma}
    Let $R$ be a ring and $M,N$ be modules. If $M \cong N$ as modules, 
    $End(M) \cong End(N)$ as rings.
\end{lemma}

In general, the converse is not true.

\newpage

\section{Direct Sums of Modules}

Blablabla.

\newpage

\section{Free Modules}

Every vector space has a basis. This is not free for modules.

Free modules are a generalization of vector spaces. They're modules that have a basis.

\begin{definition}
    A module $M$ is said to be \textbf{free} if there's a subset $S \subseteq M$ such that $S$ is a basis for $M$.
\end{definition}

\begin{definition}
    Let $R$ be a ring and $M$ be a module over $R$. $M$ is said to be \textbf{cyclic} if $M$ is generated by a single element.
\end{definition}

\begin{definition}
    A free module of rank $1$ over $R$ is called \textbf{infinite cyclic}.
\end{definition}


\begin{lemma}
    Let $R$ be a ring and $M$ be an R-module generated by $n$ elements. Then, every quotient module of $M$ can be generated by $n$ elements.
\end{lemma}
\begin{proof}

\end{proof}

\begin{corollary}
    Let $M$ be a cyclic module. Then, every quotient module of $M$ is also cyclic.
\end{corollary}

\begin{lemma}\label{cyclic_module_iso_to_r_mod_annihiliator}
    Let $M$ be a cyclic module generated by $x$. Then, 

    \[ M \cong R/\text{Ann}_{R}(x)\]
\end{lemma}
\begin{proof}
    Consider the surjective homomorphism from $R$ to $M$ given by $r \mapsto rx$.
\end{proof}


\begin{lemma}
    Let $R$ be a domain and let $M$ be a free-module of rank $n$. Then, any $n + 1$ elements of $M$ are linearly dependent, i.e. for any $y_{1}, ..., y_{n+1} \in M$ we have $r_{1},...,r_{n+1} \in R$ such that some $r_{i}$ is non-zero and

    \[ r_{1}y_{1} + ... + r_{n+1}y_{n+1} = 0\]
\end{lemma}
\begin{proof}
    If $M$ is a free-module of rank $n$, $M \cong R^{n}$. Consider the field of fractions of $R$ (call it $F$). Then, $M$ is an n-dimensional vector space. Then, there are $a_{1}, ..., a_{n+1} \in F$ such that at least one $a_{i}$ is non-zero and

    \[ a_{1}y_{1} + ... + a_{n+1}y_{n+1} = 0\]

    Clearing out the denominators of the $a_{i}$'s we can get $r_{i}$'s such that the linear combination is still 0. We thus conclude the proof.
\end{proof}

\subsection{Exercises}

\begin{lemma}\label{free_ideals_are_principal}
    Let $R$ be a commutative ring and $I \subsetneq R$ be an ideal. If $I$ is a free R-module, $I$ is principal. 
\end{lemma}
\begin{proof}
    Let $\beta$ be a finite basis for $I$. Assume by contradiction that $\beta$ has at least two elements. Let $s_{1},s_{2} \in \beta$. Then, $s_{2}s_{1} - s_{1}s_{2} = 0$, which contradicts the linear independence of $\beta$. We thus conclude the proof.
\end{proof}

\begin{lemma}
    Let $R$ be a ring. If every module over $R$ is free, $R$ is either the zero ring or a field.
\end{lemma}
\begin{proof}
    We prove the contrapositive. In other words, we prove that every ring with a non-zero non-unit ideal has a module over it that is not free. Let $R$ be a non-zero ring and $I$ be a proper ideal of $R$. Then, consider $R/I$ as an $R$-module. $R/I$ is not the zero ring, so it's not generated by the empty set. Moreover, any non-empty set $S \subseteq R/I$ is not $R$-linearly independent since multiplying by an element in the ideal maps all elements in $R/I$ to $0$. Thus, $R/I$ is not free over $R$.
\end{proof}

\begin{lemma}
    Let $R$ be a commutative ring and consider $M = Rx + Ry$ where $x,y$ are indeterminates. $M$ is not a free $R[x,y]$-module.
\end{lemma}
\begin{proof}
    Notice that $M$ corresponds to the ideal in $R[x,y]$ generated by $x$ and $y$. Therefore, to show that $M$ is not free,
    it suffices to show that the ideal $(x,y)$ in $R[x,y]$ is not principal by Lemma \ref{free_ideals_are_principal}. To see this, notice that $x$ and $y$
    are non-associate irreducibles in $R[x,y]$, so their greatest common divisor is $1$.
\end{proof}

\newpage

\section{Torsion Modules}

\begin{lemma}
    Let $R$ be a ring with zero divisors. Then, every non-zero R-module has non-zero torsion elements.
\end{lemma}
\begin{proof}
    Let $M$ be a non-zero R-module and $r$ be a zero divisor in $R$ and let $s \in R: sr = 0$. Let $x$ be a non-zero element of $M$. If $rx = 0$, we're done. Otherwise, $rx \in Tor(M)$ since $s \cdot (rx) = 0$.
\end{proof}

\begin{lemma}
    Let $R$ be a ring and $M,N$ be R-modules. Let $f: M \xrightarrow{} N$ be a module homomorphism. Then, $f(Tor(M)) \subseteq Tor(N)$.
\end{lemma}
\begin{proof}
    Let $x \in Tor(M)$. Then, there's some non-zero $r$ in $R$ such that $r \cdot x = 0$. Then, $r \cdot f(x) = f(r c\dot x) = f(0) = 0$, so $f(x) \in Tor(N)$.
\end{proof}

\begin{definition}
    A module $M$ is said to be a \textbf{torsion module} if Tor$(M) = M$.
\end{definition}

\begin{definition}
    A module $M$ is said to be \textbf{torsion-free} if $\Tor(M) = \{0\}$.
\end{definition}

\begin{lemma}
    Every finite Abelian group is a torsion $\mathbb{Z}$-module.
\end{lemma}
\begin{proof}
    Let $A$ be an Abelian group of order $n$. Considering $A$ as a $\mathbb{Z}$-module, $\forall a \in A: na = 0$. Therefore, $\forall a \in A: a \in Tor(M)$. 
\end{proof}

$\mathbb{Z}/n\mathbb{Z}[x]$ is an infinite Abelian group that's also a torsion module.


\begin{lemma}
    Let $R$ be an integral domain. Every finitely generated torsion module has a non-zero annihilator.
\end{lemma}
\begin{proof}
    Let $M$ be a finitely generated module. By the structure theorem,
\end{proof}

Here's an example that demonstrates that the finitely generated condition in the above lemma
is necessary. Let $M$ be the direct product of $\Z/p\Z$ for all primes $p$. $M$ is a torsion module,
but the annihilator of $M$ is the zero ideal.

\begin{lemma}
    Let $M$ be a module over a domain $R$. Then, $\Tor(M)$ has rank $0$.
\end{lemma}
\begin{proof}
    Notice that $\forall x \in M :\{ x \}$ is linearly dependent since there's some non-zero
    $r \in R$ such that $rx = 0$. \textit{I don't understand how to finish this.}
\end{proof}

\begin{lemma}
    Let $M$ be a module over a domain $R$. If $M$ is free, $M$ is torsion free.
\end{lemma}
\begin{proof}
    Let $B$ be a basis for $M$. By contradiction, assume there's some non-zero
    torsion element $x$ of $M$ and $r \in R: rx = 0$. Since $B$ is a basis, there's some $r_{1},...,r_{n}$ in $R$
    and $b_{1},...,b_{n}$ in $B$ such that \[ x = \sum_{i = 1}^{n} r_{i}b_{i}\]

    Then, \[ rx = \sum_{i = 1}^{n} rr_{i}b_{i}\]

    Since $R$ is a domain, $r r_{i}$ is non-zero. We have thus reached a contradiction, since we
    have a non-zero $R$-linear combination of $B$ that's zero, contradicting the $R$-linear
    independence of $B$.
\end{proof}

\begin{lemma}
    Let $M$ be a module over a domain $R$. Then, $M/\Tor(M)$ is torsion free.
\end{lemma}
\begin{proof}
    Let $x \in M$ and assume $rx \in \Tor(M)$ for some $r \in R$. Then,
    $\exists s \in R: srx = 0 \implies (sr)x = 0 \implies x \in \Tor(M)$.
\end{proof}

\begin{lemma}
    Let $R$ be a domain. Then, every ideal of $R$ (considered as a submodule)
    is torsion-free.
\end{lemma}
\begin{proof}
    
\end{proof}

\newpage

\section{Simple Modules}

\begin{definition}
    A module $M$ is said to be \textbf{simple} if $M$ is not the zero module and the only submodules of $M$ are $0$ and $M$.
\end{definition}

\begin{lemma}\label{M_cyclic_iff_any_nonzero_element_generates}
    $M$ is simple if and only if $M$ is a cyclic module generated by any nonzero element.
\end{lemma}
\begin{proof}
    Let $M$ be an simple module. Let $x$ be a non-zero element in $M$. Then, the submodule generated by $x$ has to be $M$.

    Conversely, let $M$ be a cyclic module generated by any nonzero element. Let $N$ be a nonzero submodule of $M$. Since any nonzero element is a generator, $N = M$. 
\end{proof}

\begin{example}
    Let's now try to classify all simple $\Z$-modules. Let $M$ be an simple $\Z$-module.
    Notice that the morphism from $\Z$ to $M$ defined by $x \mapsto xm$ is surjective since the image is a non-zero submodule of $M$, so it has to be $M$.
    Moreover, the kernel is not zero since $M$ can't be $\Z$, as $\Z$ is not simple. Then, $M \cong \Z/n\Z$ for some $n \in \N$.
    Then, $n$ has to be prime since otherwise we can consider Sylow subgroups and contradict irreducibility.
\end{example}


\begin{lemma}\label{simple_modules_are_quotients_of_maximal_ideals}
    Let $R$ be a commutative ring. An $R$-module $M$ is simple if and only if $M$ is isomorphic (as an R-module) to $R/I$ where $I$ is a maximal ideal of $R$.
\end{lemma}
\begin{proof}
    
\end{proof}

\begin{lemma}
    Let $M_{1},M_{2}$ be simple $R$-modules. Then, any non-zero homomorphism $f: M_{1} \xrightarrow{} M_{2}$ is an isomorphism.
\end{lemma}
\begin{proof}
    This follows immediately from the fact that $Ker(f)$ and $Im(f)$ are submodules.
\end{proof}

\begin{corollary}
    Assume $M$ is an simple $R$-module. Then, $End_{R}(M)$ is a division ring.
\end{corollary}

\newpage

\subsection{Exercises}

\begin{lemma}
    Let $F$ be a field, $R = M_{n}(F)$ and $V=F^{n}$. $V$ is a simple module over 
    $R$.
\end{lemma}
\begin{proof}
    We'll use Lemma \ref{M_cyclic_iff_any_nonzero_element_generates}. Let $0 \neq v \in V$.
    Notice that $\forall w \in V: \exists A in R: Aw = v$ simply by mapping the basis elements
    accordingly (which we can do because inverses exist). We thus conclude the proof.
\end{proof}

\newpage

\section{Finitely Generated Modules}

\begin{lemma}
    Let $M$ be an $R$-module. $M$ is finitely generated if and only if there's some
    $n \in \N$ and $\phi: R^{n} \xrightarrow{} M$ such that $\phi$ is surjective.
\end{lemma}

\begin{lemma}
    Let $M$ be a (left) R-module and $N$ be a submodule of $M$. If $N$ and $M/N$ are finitely generated, $M$ is finitely generated.
\end{lemma}
\begin{proof}
    Let $\{a_{1}, a_{2}, ..., a_{n}\}$ be a generating set for $N$ and $\{b_{1}, b_{2}, ..., b_{n}\}$ be a generating set for $M/N$. Let $f$ be the canonical surjective module homomorphism from $M$ to $M/N$. Since $f$ is surjective, for every non-zero $\hat{x} \in M/N$, there exists $x \in M$ such that $f(x) = \hat{x}$. For every $b_{i}$, pick some $c_{i}$ such that $f(c_{i}) = b_{i}$. We'll prove that $\{a_{1},a_{2},...,a_{n},c_{1},c_{2},...,c_{m}\}$ is a generating set for $M$. Let $x \in M$. We have the following two cases:

    Case 1: $x \in N$.
    Then, $x = r_{1}a_{1} + r_{2}a_{2} + ... + r_{n}a_{n}$ for some $r_{1},r_{2}, ..., r_{n}$ in $R$.

    Case 2: $x \in M - N$.
    Then, $\hat{x}$ is non-zero in $M/N$, so there are $r_{n+1}, r_{n+2}, ..., r_{m+n}$ such that $\hat{x} = r_{n+1}b_{1} + ... , + r_{m+n}b_{m}$. Pulling back using $f$, we have that $x = r_{n+1}c_{1} + ... + r_{n + m}c_{m}$.
\end{proof}

\newpage

\section{Noetherian Modules}

These are analogous to Noetherian rings.

\begin{lemma}
    Let $M$ be a Noetherian $R$-module and $f$ be a surjective endomorphism of $M$.
    Then, $f$ is an isomorphism.
\end{lemma}
\begin{proof}
    Let $M_{n} = \ker(f^{n})$. Notice that $M_{n+1} \subseteq M_{n}$ and $\forall n \in \N:
    M_{n}$ is a submodule of $M$. Since $M$ is Noetherian, $\exists n \in \N: \forall k \geq 0:
    M_{n} = M{n + k}$. 
\end{proof}

\newpage

\section{Modules over PIDs}

Modules over fields (vector spaces) are always free and we can classify finite dimensional vector spaces easily.
How much can we generalize this? It turns out, not much further than PIDs. In this scenario, the nice ideal structure 
of PIDs is reflected in the modules.

Here's some intuition for why we're considering PIDs. Let $R$ be a domain. Recall that $R$ is a module
over itself. Then, every ideal of $R$ is a submodule. Recall that if $I$ is a free submodule, $I$ is principal.
Then, if every submodule of $R$ is free, $R$ is a PID.

\begin{theorem}
    Let $M$ be a module over a PID $R$. If $M$ is free, every submodule of $M$ is free.
\end{theorem}
\begin{proof}
    
\end{proof}

This result doesn't hold when $R$ is not a PID.
In fact, any non-principal ideal can be used to construct a counterexample. 
Let $R$ be a ring and $I$ be a non-principal ideal in $R$. 
Then, $I$ is not free over $R$ since every $I$ that is free over $R$ is principal by Lemma \ref{free_ideals_are_principal}.

\begin{lemma}\label{injective_presentations_exist}
    Every finitely generated $R$-module $M$ is isomorphic to $R^{n}/Im(g)$ for some $g: R^{m} \xrightarrow{} R^{n}$.
\end{lemma}
\begin{proof}
    Let $M$ be an $R$-module generated by $n$ elements.
    Then, there's a surjective homomorphism $f:R^{n} \xrightarrow{} M$.
    Since $\ker(f)$ is a submodule of $R^{n}$, $\ker(f) \cong R^{m}$ for some $m \leq n$. Let $\phi$ be the isomorphism.
    Then, let $g$ be the map defined by applying $\phi$ and injecting $\ker(f)$ into $R^{n}$. Notice that $Im(g) = \ker(f)$,
    so $M \cong R^{n}/Im(g)$.
\end{proof}

\begin{definition}
    A \textbf{presentation of M} is a homomorphism $g: R^{m} \xrightarrow{} R^{n}$ and an isomorphism $M \cong R^{n}/Im(g)$.
\end{definition}

Notice that using Lemma \ref{injective_presentations_exist}, we can ensure that $g$ is injective.
However, this isn't necessary.

Let $A \in M_{nxm}(R)$ such that $Ax = g(x)$. Such an $A$ exists since $g$ is 
a homomorphism of free modules. We can thus represent any finitely generated module $M$
using a matrix. This presentation is not unique. We can analyze the matrix representation of finitely
generated modules in order to learn more about the module itself.

\subsection{Structure Theorem of Finitely Generated Modules over PIDs}

\begin{corollary}
    Let $M$ be a finitely generated module over a PID $R$. M is free if and only if
    $M$ is torsion free.
\end{corollary}
\begin{proof}
    You have to generate the torsion element, but you can't generate it without being
    linearly dependent.
\end{proof}

\begin{lemma}
    Let $M$ be a finitely generated torsion module over a PID $R$.
    Then, $M$ is a simple module if and only if $M = \langle x \rangle$ and $Ann(x) = (p)$
    for some prime $p$.
\end{lemma}
\begin{proof}
    Applying the structure theorem for finitely generated modules, we get that
    $M \cong R/I$ for some ideal $I$ of $R$. Then, using Lemma 
    \ref{simple_modules_are_quotients_of_maximal_ideals}
    , $I$ has to be a maximal ideal. Since prime ideals are maximal ideals in PIDs, we conclude the proof.
\end{proof}

\newpage

\section{Modules over Euclidean Domains}

There's an algorithmic version of the theorems we've proven for PIDs. They're presented here.

\begin{theorem}[Existence of Invariant Factor Form]
    
\end{theorem}
\begin{proof}
    Let $M$ be a module and let $g: R^{m} \xrightarrow{} R^{n}$ such that $M \cong R^{n}/Im(g)$. We proved that there's invertible $h: R^{m} \xrightarrow{} R^{m}$ and $f: R^{n} \xrightarrow{} R^{n}$ such that $M \cong R^{n}/Im(fgh)$. Since the matrix corresponding to $fgh$ is in normal form, $Im(fgh) = R^{s} \bigoplus d_{1}R \bigoplus d_{2}R ... \bigoplus d_{k}R$. We therefore conclude the proof.
\end{proof}

\begin{corollary}
    Let $R$ be a PID. Every torsion free finitely generated R-module is free.
\end{corollary}

The elementary divisors of a finitely generated module $M$ are just the invariant factors of the primary components of Tor$(M)$.

\begin{theorem}[Existence of Elementary Divisor Form]

    
\end{theorem}
\begin{proof}
    
\end{proof}

\begin{lemma}
    Let $R$ be a PID and $M$ be an R-module. $M$ is cyclic if and only if $M \cong R/(a)$ for some $a \in R$.
\end{lemma}
\begin{proof}
    Assume $M$ is cyclic. Then, there's some $x \in M$ such that $x$ generates $M$. Consider the module homomorphism $\phi_{x}: R \xrightarrow{} M$ given by $\phi(r) = rx$. Since $x$ generates $M$, $\phi$ is surjective. Since $R$ is a PID, $ker(\phi_{x}) = (a)$ for some $a \in R$.
    
    Then, by the first isomorphism theorem for modules, $M \cong R/(a)$. 

    For the converse, notice that $a$ is a generator for the module $R/(a)$.
\end{proof}
\begin{proof}
    One can also consider Lemma \ref{cyclic_module_iso_to_r_mod_annihiliator}. The fact that $R$ is a PID immediately gives us the desired result.
\end{proof}

\begin{corollary}
    Let $R$ be a PID and $M$ be a cyclic R-module. Then, every submodule of $M$ is also cyclic.
\end{corollary}
\begin{proof}
    Let $M$ be a cyclic module over a PID R. Then, $M \cong R/aR$ for some $a \in R$. Let $N$ be a submodule of $M$. Then, $N$ is an ideal of $R/aR$. Recall that every ideal in the ring $R/aR$ corresponds to an ideal in the ring $R$ that contains $aR$. Since $R$ is a PID, every ideal in $R/aR$ is also principal. Then, there's a single element that generates $N$.
\end{proof}

\begin{lemma}
    Let M be a finitely generated torsion module over a PID R and let $n = \lvert IF(M) \rvert$. M can be generated by n elements and can't be generated by less than n elements.
\end{lemma}
\begin{proof}
    Let M be a finitely generated torsion module over a PID R and let $n = \lvert IF(M) \rvert$. Then, 

    \[ M \cong R/d_{1}R \oplus ... R/d_{n}R\]

    for some $d_{i} \in R$ such that $d_{i} \mid d_{i + 1}$. Notice that the set $\{e_{1},...,e_{n}\}$ generates the right hand side. 
    
    We'll now prove that $M$ can't be generated by $n-1$ elements. Assume by contradiction that $M$ can be generated using $m < n$ elements. Then, $M \cong R^{m}/N$ where $N$ is a submodule of $R^{m}$. However, this immediately implies that $M$ has at most $m$ invariant factors, which is a contradiction.
\end{proof}

\newpage

\section{Finitely Generated Abelian Groups}

Recall that there's a one-to-one correspondence between Abelian groups and $\Z$-modules. 

Also recall that we can identify every ideal of $\Z$ with a unique positive integer $d$.

\subsection{Module of Fractions, Limited Construction}

Let $R$ be a domain and $F$ be its field of fractions.

We'll put an equivalence relation on $M \times R$ by letting $(m,a) \sim (m^{\prime}, a^{\prime})$ if and only if $am^{\prime} = a^{\prime}m$.

We'll denote by $FM$ the set of equivalence classes.

\begin{lemma}
    $FM$ is a vector space over $F$.
\end{lemma}
\begin{proof}
    
\end{proof}

Notice that we've defined $\phi$ that goes from $R$-modules to $F$-modules. $\phi$ is called a functor and also acts on module homomorphisms. More formally, if $g: M \xrightarrow{} N$, $Fg: FM \xrightarrow{} FN$ where $\phi(g) = Fg$.

\subsection{Determination of rank by using reductions to linear algebra}

Let $R$ be a PID and $M$ be a finitely generated $R$-module. Recall that $M \cong Tor(M) \oplus R^{s}$ where $s = rank(M)$. 

\subsection{Exercises}

If $M$ is a finite Abelian group, $M$ is naturally a $\Z$-module. Can this action
be extended to make $M$ into a $\Q$-module?



\newpage

\section{Modules over F[x]}

Sherstov: It is hard to factor integers. However, it is really easy to factor polynomials over fields into irreducibles, because polynomials have a rich algebraic structure.

Let $F$ be a field and $R = F[x]$. Recall that $R$ is a Euclidean Domain. In particular, $R$ is a PID. Recall that we can identify every ideal of $R$ with a unique monic polynomial in $F[x]$.

Also recall that $F$ is a subring of $R$ corresponding to constant polynomials. This is useful because every $R$-module $M$ can be made into a vector space over $F$ by considering the module structure given by the pullback. Equivalently, this can be thought of as restricting the scalars.

Let $0 \neq g \in R$ be a non-constant polynomial. Our goal is now to understand $R/gR$ better. $R/gR$ is a module over $R$, so we can use the trick in mentioned in the previous paragraph and find $\dim_{F}(R/gR)$.

We'll think of $F$ as being contained in $R/gR$. More formally, here's how we injectively map $F$ into $R/gR$ as a ring. Consider the canonical ring homomorphism from $R \xrightarrow{} R/gR$ defined by $g \mapsto \bar{g}$. Then, consider the map $F \xhookrightarrow{} R \xrightarrow{} R/gR$. This is a composition of ring homomorphisms, so it's a ring homomorphism. Since the domain is a field, it has to be injective. We'll just denoted $\hat{a} \in R/gR$ by $a$ and assume $F$ is contained in $R/gR$ for notational ease.

\begin{proposition}
    \( \dim_{F}(R/gR) = n = deg(g)\)
\end{proposition}
\begin{proof}
    Let $g = a_{0} + a_{1}x + ... + x^{n}$. Notice that $g$ is monic since we're identifying $R/gR$ with its unique monic polynomial. 
    
    We'll now show that \( \{ \bar{1}, \bar{x}, ..., \bar{x}^{n-1}\}\) generate $R$.

    Let $h \in R$. Then, $h = g \cdot q + r$ for some unique $q,r \in R$ such that $r = 0$ or $deg(r) < n$. Notice that $\bar{h} = bar{r}$. If $r = 0$, $\bar{h} = 0$. Otherwise, $\bar{h}$ is an F-linear combination of  \( \{ \bar{1}, \bar{x}, ..., \bar{x}^{n-1}\}\) since $r$ is a linear combination of  \( \{ {1}, {x}, ..., {x^{n-1}}\}\). We have thus proven that \( \{ \bar{1}, \bar{x}, ..., \bar{x}^{n-1}\}\) is a generating set. We'll now prove that it's independent.

    Assume that \( \sum_{i = 0}^{n-1} a_{i}\bar{x}^{i} = 0\) in $R/gR$. Then,  \( f = \sum_{i = 0}^{n-1} a_{i}{x}^{i} \in gR\). Then, $f$ is a polynomial of degree less than $n$ divisible by a polynomial of degree $n$, so $f = 0$. We thus conclude the proof.
\end{proof}

Now, let $R$ be a polynomial ring over a field $M$ be a finitely generated $R$-module. Then, by the invariant form for finitely generated modules, we have that

\[ M \cong R/f_{1}R \oplus R/f_{2}R \oplus ... \oplus R/f_{r}R \oplus R^{s} \]

Then, by the proposition,

\[ \text{M is torsion} \iff s = 0 \iff \dim_{F}(M) \text{ is finite}\]

In mathematics, there are sometimes different languages to describe the same phenomenon. In this case, we'll have 3 different languages to describe the same phenomenon.

\section{Indecomposable Modules}

\begin{definition}
    A module is called \textbf{indecomposable} if it can't be expressed as a direct sum of its submodules.
\end{definition}

\newpage

\section{Canonical Forms using IF and ED Form}

Canonical forms are useful since they give us a method to test if
two linear operators are identical. In other words, it let's us check
whether two matrices are similar.

This is another case where we use the structure of the space being acted upon is used to obtain
information on the algebraic objects which are acting.

\subsection{The Characteristic and Minimal Polynomial}

\begin{definition}
    The polynomial $\det(xI - T)$ is called the \textbf{characteristic polynomial of $T$} and
    is denoted $c_{T}(x)$.
\end{definition}

\begin{definition}
    The unique monic polynomial which generates $Ann(V)$ in $F[x]$ is called the \textbf{minimal polynomial of $T$}
    and is denoted $m_{T}(x)$. 
\end{definition}

Notice that the definition also implies that $m_{T}(x)$ divides any polynomial $f$ with $f(T) = 0$.

\begin{proposition}
    The minimal polynomial $m_{T}(x)$ is the largest invariant factor of $V$. All the invariant factors
    divide $m_{T}(x)$.
\end{proposition}
\begin{proof}
    $F[x]$ is a PID and the last invariant factor generates the annihilator since all the other IFs
    divide it and it annihilates the last cyclic module.
\end{proof}

\begin{theorem}[Cayley-Hamilton]
    The minimal polynomial for $T$ divides the characteristic polynomial of $T$.
\end{theorem}
\begin{proof}
    Since the characteristic polynomial is the product of all invariant factors, this is immediate.
\end{proof}

\begin{lemma}
    Let $A \in M_{n}(F)$. The minimal polynomial of $A$ has the same irreducible divisors
    as the characteristic polynomial of $A$.
\end{lemma}
\begin{proof}
    Since the minimal polynomial divides the characteristic polynomial, it suffices to show
    that every irreducible divisor of the characteristic polynomial divides the minimal polynomial.

    Let $f $ be an irreducible polynomial that divides the characteristic polynomial. Then, $f$ is
    also prime since $F[x]$ is a Euclidean domain. Then, $f$ divides one of the invariant factors by
    a simple inductive argument on the number of invariant factors. Since the minimal polynomial is
    the greatest invariant factor, $f$ also divides the minimal polynomial.
\end{proof}

\begin{corollary}
    The characteristic polynomial divides a power of the minimal polynomial.
\end{corollary}

\subsection{Rational Canonical Form}

The rational canonical form doesn't concern itself with finding a nice representation of the matrix.
It just finds a \textbf{canonical} representation, however ugly it is.

Let $V$ be a finite dimensional vector space over $F$ with dimension $n$. 

Any nonzero free $F[x]$-module (being isomorphic to a direct sum of copies of $F[x]$)
is an infinite dimensional vector space over $F$. Since $V$ is finite dimensional,
$V$ is a torsion $F[x]$-module.

\begin{lemma}
    Let $a(x) \in F[x]$. The characteristic polynomial of the companion matrix of $a(x)$ is $a(x)$.
\end{lemma}




\begin{theorem}[Rational Canonical Form]
    Let $A \in M_{n}(F)$. Then, $A$ is conjugate (similar) to 

    \[ \]

    for unique monic polynomials $f_{1} \mid f_{2} \mid ... \mid f_{r}$.
\end{theorem}

\begin{corollary}
    $A,B$ in $M_{n}(F)$ are similar if and only if $RFC(A) = RFC(B)$.
\end{corollary}

We found representatives of conjugacy classes in order to check similarity.

\begin{theorem}[Rational Canonical Form for Linear Operators]
    Let $A$ be a linear operator in a finite dimenasional vector space
    over $F$. Then, there's a basis such that [A] is in rational canonical
    form.
\end{theorem}

The rational canonical form of a linear operator stays the same in the $K$ if $F$ is
a subfield of $K$.

\subsection{Jordan Canonical Form}

We'll assume that $F$ contains all eigenvalues of $T$. In other words, we assume that the characteristic
polynomial of $T$ splits over $F$.

\begin{proposition}
    The following conditions are equivalent:

    \begin{enumerate}
        \item 
    \end{enumerate}
\end{proposition}

\begin{corollary}
    $T$ is diagonalizable if and only if $m_{T}(x)$ doesn't have repeated roots.
\end{corollary}
\begin{proof}
    
\end{proof}

Jordan forms are not useful numerically.

\newpage

\subsection{Exercises}

\begin{lemma}
    Let $A \in M_{n}(R)$ such that $A = A^{2}$. Then, $A$ is diagonalizable
    with 0s and 1s on the diagonal.
\end{lemma}
\begin{proof}
    Notice that the minimal polynomial of $A$ divides $x^{2} - x = x (x-1)$. Then,
    all the invariant factors are $x(x-1), x, (x-1)$. Then, the characteristic
    polynomial splits and $0$ and $1$ are the only roots of the characteristic polynomial.
\end{proof}

\newpage

\section{Exact Sequences: Projective, Injective and Flat Modules}

Given two modules $A,C$, how many possible ways are there for us to extend
$C$ to $B$ such that $B/A \cong C$? Is this always possible?

Let $A,C$ be $R$-modules. Then, letting $B := C \oplus A$, $B/A \cong C$.
Therefore, $1 \xrightarrow{} A \xrightarrow{} C \oplus A \xrightarrow{} C \xrightarrow{} 1$
is a short exact sequence.

Notice that if $A,C$ are groups, this means that $B$ is the semidirect product
of $A$ and $C$. This is another way of seeing that semidirect products are direct
sums for Abelian groups.

\begin{definition}
    Let $0 \xrightarrow{} A \xrightarrow{\psi} B \xrightarrow{\phi} C \xrightarrow{} 0$ be a 
    short exact sequence of $R$-modules. The sequence is said to be \textbf{split} if there's a map
    $\alpha: C \xrightarrow{} B$ such that $\phi \circ \alpha$ is the identity map on $C$.
\end{definition}



\subsection{Exercises}

\begin{lemma}
    Let $R$ be a ring. Prove that every $R$-module is injective if and only if
    every $R$-module is projective.
\end{lemma}
\begin{proof}
    
\end{proof}

\end{document}
